\input{preambuloSimple.tex}

%----------------------------------------------------------------------------------------
%	TÍTULO Y DATOS DEL ALUMNO
%----------------------------------------------------------------------------------------

\title{	
\normalfont \normalsize 
\textsc{\textbf{Ingeniería de Servidores (2016-2017)} \\ Grado en Ingeniería Informática \\ Universidad de Granada} \\ [25pt] % Your university, school and/or department name(s)
\horrule{0.5pt} \\[0.4cm] % Thin top horizontal rule
\huge Memoria Práctica 5 \\ % The assignment title
\horrule{2pt} \\[0.5cm] % Thick bottom horizontal rule
}

\author{Cristian Vélez Ruiz} % Nombre y apellidos

\date{\normalsize\today} % Incluye la fecha actual

%----------------------------------------------------------------------------------------
% DOCUMENTO
%----------------------------------------------------------------------------------------
\begin{document}

\maketitle % Muestra el Título

\newpage %inserta un salto de página

\tableofcontents % para generar el índice de contenidos

\listoffigures

\listoftables

\newpage

%----------------------------------------------------------------------------------------
%	Cuestión 1
%----------------------------------------------------------------------------------------
\section[Cuestión 1]{Al modificar los valores del kernel de este modo, no logramos que persistan después de reiniciar la máquina. ¿Qué archivo hay que editar para que los cambios sean permanentes?}

Para que los cambios que realicemos con sysctl sean persistentes, necesitamos modificar el archivo \textbf{/etc/sysctl.conf} \cite{kernel}, donde se guarda la configuración del kernel y se cargará una vez que ha arrancado nuestro sistema.

En mi caso modificaré la variable para que cuando haya un kernel panic se reinicie a los 10 segundos.

\begin{itemize}
	\item Observamos el estado actual de la variable con \textit{sysctl -a | grep panic}:
	
	\begin{figure}[H] %con el [H] le obligamos a situar aquí la figura
		\centering
		\includegraphics[scale=0.5]{pics/kernel}  %el parámetro scale permite agrandar o achicar la imagen. En el nombre de archivo puede especificar directorios
		\caption{Estado kernel panic} \label{fig:kernel1}
	\end{figure}
	Como se puede apreciar esta configurado a 0.
	
	\item Ahora para cambiarlo a 10, debemos hacer \textit{sudo sysctl -w kernel.panic=10}.
	
	\begin{figure}[H] %con el [H] le obligamos a situar aquí la figura
		\centering
		\includegraphics[scale=0.5]{pics/kernel1}  %el parámetro scale permite agrandar o achicar la imagen. En el nombre de archivo puede especificar directorios
		\caption{Cambiando kernel panic} \label{fig:kernel2}
	\end{figure}

	\item Una vez modificado debemos hacerlo persistente añadiendo al archivo \textbf{/etc/sysctl.conf} con \textbf{echo "kernel.panic = 10" >> /etc/sysctl.conf}.

\item Una vez realizado actualizamos la configuración de sysctl con \textit{sysctl -p}.

\begin{figure}[H] %con el [H] le obligamos a situar aquí la figura
	\centering
	\includegraphics[scale=0.5]{pics/kernel2}  %el parámetro scale permite agrandar o achicar la imagen. En el nombre de archivo puede especificar directorios
	\caption{Cambiando kernel panic} \label{fig:kernel3}
\end{figure}

Como se puede apreciar, se ha actualizado kernel.panic y ya es permanente.
\end{itemize}


	
	

%----------------------------------------------------------------------------------------
%	Cuestión 2
%----------------------------------------------------------------------------------------
\section[Cuestión 2]{¿Con qué opción se muestran todos los parámetros modificables en tiempo de ejecución? Elija dos parámetros y explique, en dos líneas, qué función tienen.}

Para mostrar los parámetros en tiempo de ejecución, se usa \textbf{sysctl -a} \cite{kernel} \cite{sysctl}

\begin{figure}[H] %con el [H] le obligamos a situar aquí la figura
	\centering
	\includegraphics[scale=0.5]{pics/a}  %el parámetro scale permite agrandar o achicar la imagen. En el nombre de archivo puede especificar directorios
	\caption{sysctl -a} \label{fig:kernel4}
\end{figure}

\begin{itemize}
	\item \textbf{kernel.panic\_on\_stackoverflow} : Si se produce un desbordamiento de pila y esta configurado a 0, intentará continuar con la operación, en cambio si esta a 1 entrará en modo pánico.
	
	\item \textbf{kernel.pid\_max} :Cuando el siguiente PID del kernel alcanza este valor, se ajusta a un valor mínimo de PID, ni el mínimo ni el máximo se utilizan.
\end{itemize}


%----------------------------------------------------------------------------------------
%	Cuestión 3
%----------------------------------------------------------------------------------------
\section[Cuestión 3]{a) Realice una copia de seguridad del registro y restáurela, ilustre el proceso con capturas. b) Abra una ventana mostrando el editor del	registro.}

Para realizar la copia debemos ir a regedit (editor de registros) \cite{regedit}:

\begin{figure}[H] %con el [H] le obligamos a situar aquí la figura
	\centering
	\includegraphics[scale=0.5]{pics/reg}  %el parámetro scale permite agrandar o achicar la imagen. En el nombre de archivo puede especificar directorios
	\caption{Copia seguridad regedit} \label{fig:reg}
\end{figure}

Una vez abierto damos a \textit{exportar}:

\begin{figure}[H] %con el [H] le obligamos a situar aquí la figura
	\centering
	\includegraphics[scale=0.5]{pics/reg1}  %el parámetro scale permite agrandar o achicar la imagen. En el nombre de archivo puede especificar directorios
	\caption{Copia seguridad regedit 1} \label{fig:reg1}
\end{figure}

Añadimos el nombre a la copia de seguridad y donde queremos guardarla:

Una vez abierto damos a exportar:

\begin{figure}[H] %con el [H] le obligamos a situar aquí la figura
	\centering
	\includegraphics[scale=0.5]{pics/red2}  %el parámetro scale permite agrandar o achicar la imagen. En el nombre de archivo puede especificar directorios
	\caption{Copia seguridad regedit 2} \label{fig:reg2}
\end{figure}

Ya tenemos nuestra copia de seguridad realizada.

\begin{figure}[H] %con el [H] le obligamos a situar aquí la figura
	\centering
	\includegraphics[scale=0.5]{pics/reg3}  %el parámetro scale permite agrandar o achicar la imagen. En el nombre de archivo puede especificar directorios
	\caption{Copia seguridad regedit 3} \label{fig:reg3}
\end{figure}

Para restaurar la copia de seguridad, vamos a \textit{importar}:

\begin{figure}[H] %con el [H] le obligamos a situar aquí la figura
	\centering
	\includegraphics[scale=0.5]{pics/reg4}  %el parámetro scale permite agrandar o achicar la imagen. En el nombre de archivo puede especificar directorios
	\caption{Copia seguridad regedit 4} \label{fig:reg4}
\end{figure}

Seleccionamos la copia de seguridad a restaurar:

\begin{figure}[H] %con el [H] le obligamos a situar aquí la figura
	\centering
	\includegraphics[scale=0.5]{pics/reg5}  %el parámetro scale permite agrandar o achicar la imagen. En el nombre de archivo puede especificar directorios
	\caption{Copia seguridad regedit 5} \label{fig:reg5}
\end{figure}

Seleccionamos \textit{abrir} y comenzará el proceso de restauración.

\begin{figure}[H] %con el [H] le obligamos a situar aquí la figura
	\centering
	\includegraphics[scale=0.5]{pics/reg6}  %el parámetro scale permite agrandar o achicar la imagen. En el nombre de archivo puede especificar directorios
	\caption{Copia seguridad regedit 6} \label{fig:reg6}
\end{figure}

Una vez finalice, ya tenemos restaurada la copia de seguridad.



%----------------------------------------------------------------------------------------
%	Cuestión 4
%----------------------------------------------------------------------------------------
\section[Cuestión 4]{ Enumere qué elementos se pueden configurar en Apache y en IIS para que Moodle funcione mejor.}

En Apache algunos de los elementos que podemos configurar para moodle \cite{moodle} son:

\begin{itemize}
	\item MaxClients para evitar que el sistema se quede sin memoria.
	\item Reducir el numero de módulos que Apache carga desde httpd.conf para reducir el uso de memoria.
	\item Disminuir MaxRequestsPerChild a 20-30.
	\item Configurar un Reverse Proxy server para reducir la carga del servidor para la descarga de archivos con imágenes
	\item Configurar el Directory Index DirectoryIndex correctamente para evitar la negociación de contenido.
	\item Desactivar HostnameLookups para reducir la latencia del dns.
	\item Reducir el TimeOut a 30-60.
	
\end{itemize}

En ISS algunos de los elementos que podemos configurar son:

\begin{itemize}
	\item ListenBackLog es el equivalente de KeepAliveTimeout establecer entre 2-5.
	\item MemCacheSize tamaño de la memoria cache, por defecto esta configurado a la mitad disponible.
	\item ObjectCacheTTL tiempo en el que los objetos permanecen en la memoria cache.
	
	
\end{itemize}

%----------------------------------------------------------------------------------------
%	Cuestión 5
%----------------------------------------------------------------------------------------
\section[Cuestión 5]{ Ajuste la compresión en el servidor y analice su comportamiento usando varios valores para el tamaño de archivo a partir del cual comprimir. Para comprobar que está comprimiendo puede usar el navegador o comandos como curl (see url) o lynx. Muestre capturas de pantalla de todo el proceso}

Para poder comprobar la compresión del servidor voy a modificar la pagina por defecto del servidor como hicimos en la practica anterior, para que tenga un tamaño considerable para ser comprimida, casi 2KiB.

\begin{figure}[H] %con el [H] le obligamos a situar aquí la figura
	\centering
	\includegraphics[scale=0.5]{pics/2}  %el parámetro scale permite agrandar o achicar la imagen. En el nombre de archivo puede especificar directorios
	\caption{Archivo por defecto} \label{fig:1}
\end{figure}

Una vez realizado vamos a la configuración de IIS y después al apartado \textit{Compresión} y habilitamos las pestañas de compresión estática, y también configuramos sus parámetros.

\begin{figure}[H] %con el [H] le obligamos a situar aquí la figura
	\centering
	\includegraphics[scale=0.5]{pics/1}  %el parámetro scale permite agrandar o achicar la imagen. En el nombre de archivo puede especificar directorios
	\caption{Compresión del servidor} \label{fig:2}
\end{figure}

\begin{figure}[H] %con el [H] le obligamos a situar aquí la figura
	\centering
	\includegraphics[scale=0.5]{pics/3}  %el parámetro scale permite agrandar o achicar la imagen. En el nombre de archivo puede especificar directorios
	\caption{Compresión del servidor 2} \label{fig:3}
\end{figure}

Desde otra maquina lanzaré la orden \textit{curl -I -H 'Accept-Encoding:gzip,deflate'} antes de aplicar los cambios y una vez aplicados para ver el tamaño de la  respuesta a la petición.

\begin{figure}[H] %con el [H] le obligamos a situar aquí la figura
	\centering
	\includegraphics[scale=0.5]{pics/4}  %el parámetro scale permite agrandar o achicar la imagen. En el nombre de archivo puede especificar directorios
	\caption{Compresión del servidor 3} \label{fig:4}
\end{figure}

Como se puede apreciar la longitud de la respuesta ha pasado de 2264167 a 570363, lo que es casi un cuarto de su longitud anterior.



%----------------------------------------------------------------------------------------
%	Cuestión 6
%----------------------------------------------------------------------------------------
\section[Cuestión 6]{Usted parte de un SO con ciertos parámetros definidos en la	instalación (Práctica 1), ya sabe instalar servicios (Práctica 2) y cómo monitorizarlos (Práctica 3) cuando los somete a cargas (Práctica 4). Al igual que ha visto cómo se puede mejorar un servidor web (Práctica 5 Sección 3.1), elija un servicio (el que usted quiera) y modifique un parámetro para	mejorar su comportamiento. 6.b) Monitorice el servicio antes y después de la modificación del parámetro aplicando cargas al sistema (antes y después) mostrando los resultados de la monitorización.}

\begin{enumerate}[label=(\alph*)]
	\item 
Modificaré en el servicio Apache dos parámetros:

\begin{itemize}
	\item MaxKeepAliveRequest: Número máximo de peticiones a permitir en una conexión persistente. Antes estaba: 100 , ahora: 64
	
	\item KeepAliveTimeout: Número de segundos a esperar a la siguiente petición de algún cliente. Antes estaba: 5 , ahora: 3
\end{itemize}

Antes de que sea modificado se hará la prueba del ejercicio 6.b.\\

Para modificar estos datos debemos ir a /etc/apache2/apache2.conf, hacemos una copia de seguridad, seguidamente editamos los valores y reiniciamos el servicio.

\begin{figure}[H] %con el [H] le obligamos a situar aquí la figura
	\centering
	\includegraphics[scale=0.5]{pics/12}  %el parámetro scale permite agrandar o achicar la imagen. En el nombre de archivo puede especificar directorios
	\caption{Optimizando Apache} \label{fig:5}
\end{figure}

\begin{figure}[H] %con el [H] le obligamos a situar aquí la figura
	\centering
	\includegraphics[scale=0.5]{pics/13}  %el parámetro scale permite agrandar o achicar la imagen. En el nombre de archivo puede especificar directorios
	\caption{Optimizando Apache 1} \label{fig:6}
\end{figure}

\item Para obtener monitorizar el rendimiento usaré Apache Benchmark como en la práctica anterior con ab -n 50000 -c 50 http://direccion/index.html, antes y después de la modificación.

\begin{itemize}
	\item \textbf{Configuración por defecto:}
	
	\begin{figure}[H] %con el [H] le obligamos a situar aquí la figura
		\centering
		\includegraphics[scale=0.5]{pics/111}  %el parámetro scale permite agrandar o achicar la imagen. En el nombre de archivo puede especificar directorios
		\caption{Configuración por defecto} \label{fig:7}
	\end{figure}
	
	\item \textbf{Configuración modificada:}
	
		\begin{figure}[H] %con el [H] le obligamos a situar aquí la figura
		\centering
		\includegraphics[scale=0.5]{pics/14}  %el parámetro scale permite agrandar o achicar la imagen. En el nombre de archivo puede especificar directorios
		\caption{Configuración modificada} \label{fig:8}
	\end{figure}
	
\end{itemize}

Como podemos apreciar con la configuración personalizada hemos reducido el tiempo medio por petición en 1 ms.

\end{enumerate}



\bibliography{citas} %archivo citas.bib que contiene las entradas 
\bibliographystyle{plain} % hay varias formas de citar

\end{document}

\grid
